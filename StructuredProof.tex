\documentclass[letterpaper,twoside,10pt]{report}

%% Language %%%%%%%%%%%%%%%%%%%%%%%%%%%%%%%%%%%%%%%%%%%%%%%%%
\usepackage[USenglish]{babel} %francais, polish, spanish, ...
\usepackage[T1]{fontenc}
\usepackage[ansinew]{inputenc}
\usepackage{lmodern} %Type1-font for non-english texts and characters

%% Packages for Graphics & Figures %%%%%%%%%%%%%%%%%%%%%%%%%%
\usepackage{graphicx} %%For loading graphic files

%% Math Packages %%%%%%%%%%%%%%%%%%%%%%%%%%%%%%%%%%%%%%%%%%%%
\usepackage{amsmath}
\usepackage{amsthm}
\usepackage{amsfonts}


\begin{document}

\pagestyle{empty} %No headings for the first pages.


\pagestyle{plain} %Now display headings: headings / fancy / ...

Proof. The square of any odd number is also odd. 

\begin{enumerate} 
	%Step 1
	\item Let $x = 2a + 1$, where $a \in \mathbb{Z}$
		%Proof of step 1
		\begin{enumerate}	
			\item By definition, an odd number is a number which is not even. An even number can be represented as $2k$ for some $k \in \mathbb{Z}$. If a number is not even, then it can be represented in the form $2k'+1$ where $k'\in\mmathbb{Z}$.
			%Proof of the representation of an odd number
			\begin{enumerate}	
				\item Assume for a contradiction $\exists k\in\mathbb{Z}$ such that $2k = x$, and that $x=2k'+1$for some $k \in \mathbb{Z}$. Then:
				\begin{align*}
				2k &= 2k'+1\\
				k &= k'+\dfrac{1}{2}
				\end{align*}
				But for $k'\in\mathbb{Z}$, then $k'+ \dfrac{1}{2}$ is not also an integer. QED.
				\end{enumerate}
		\end{enumerate}
	%Step 2
	\item $x^2 = 2a' + 1$, where $a' = (2a^2 + 2a)$
		%Proof of step 2
		\begin{enumerate}
			\item Multiply each term and add like terms (a.k.a. "FOIL" or "First, outer, inner, last"): 
				\begin{align*}
					x^2 &= x \cdot x \\
					&= (2a+1)(2a+1)\\
					&= (2a)(2a) + (2a)(1) + (2a)(1) + (1) \\
					&= 4a^2 + 2a + 2a + 1 \\
					&= 4a^2 + 4a + 1 \\
					&= 2(2a^2 + 2a) + 1 \\
					&= 2a' + 1 \text{ where } a' = 2a^2 + 2a
				\end{align*}
		\end{enumerate}
	%Step 3
	\item QED
		\begin{enumerate}
			\item need to prove that $a' \in \mathbb{Z}$
			\item By corollary 2, the converse of corollary 1 is also true. That is, any number which can be represented as $2a + 1$, where $a\in \mathbb{N}$, is considered odd.  
		\end{enumerate}
\end{enumerate}

Corollary 1. Any odd number can be represented in the form $2a+1$ where $a \in \mathbb{N}$.
An odd number is, by definition, a number which is not even. An even number can be represented as $2k$ for some $k \in \mathbb{Z}$. 

%% Chapters %%%%%%%%%%%%%%%%%%%%%%%%%%%%%%%%%%%%%%%%%%%%%%%%%
%% ==> Write your text here or include other files.

%\input{intro} %You need a file 'intro.tex' for this.


%%%%%%%%%%%%%%%%%%%%%%%%%%%%%%%%%%%%%%%%%%%%%%%%%%%%%%%%%%%%%
%% ==> Some hints are following:

%\chapter{Some small hints}\label{hints}

%\section{German Umlauts and other Language Specific Characters}\label{umlauts}
%You can type german umlauts like '�', '�', or '�' directly in this file.
%This is also true for other language specific characters like '�', '�' etc.
%
%There are problems with automatic hyphenation when using language
%specific characters and OT1-encoded fonts. In this case, use a
%T1-encoded Type1-font like the Latin Modern font family (\verb#\usepackage{lmodern}#).
%
%
%\section{References}\label{references}
%Using the commands \verb#\label{name}# and \verb#\ref{name}# you are able
%to use references in your document. Advantage: You do not need to think
%about numerations, because \LaTeX\ is doing that for you.
%
%For example, in section \ref{dividing} on page \pageref{dividing} hints for
%dividing large documents are given.
%
%Certainly, references do also work for tables, figures, formulas\ldots
%
%Please notice, that \LaTeX\ usually needs more than one run (mostly 2) to
%resolve those references correctly.
%
%
%\section{Dividing Large Documents}\label{dividing}
%You can divide your \LaTeX-Document into an arbitrary number of \TeX-Files
%to avoid too big and therefore unhandy files (e.g. one file for every chapter).
%
%For this, you insert in your main file (this one) for every subfile
%the command '\verb#\input{subfile}#'. This leads to the same behavior
%as if the content of the subfile would be at the place of the \verb#\input#-Command.

%% <== End of hints
%%%%%%%%%%%%%%%%%%%%%%%%%%%%%%%%%%%%%%%%%%%%%%%%%%%%%%%%%%%%%



%%%%%%%%%%%%%%%%%%%%%%%%%%%%%%%%%%%%%%%%%%%%%%%%%%%%%%%%%%%%%
%% BIBLIOGRAPHY AND OTHER LISTS
%%%%%%%%%%%%%%%%%%%%%%%%%%%%%%%%%%%%%%%%%%%%%%%%%%%%%%%%%%%%%
%% A small distance to the other stuff in the table of contents (toc)
%\addtocontents{toc}{\protect\vspace*{\baselineskip}}
%
%%% The Bibliography
%%% ==> You need a file 'literature.bib' for this.
%%% ==> You need to run BibTeX for this (Project | Properties... | Uses BibTeX)
%%\addcontentsline{toc}{chapter}{Bibliography} %'Bibliography' into toc
%%\nocite{*} %Even non-cited BibTeX-Entries will be shown.
%%\bibliographystyle{alpha} %Style of Bibliography: plain / apalike / amsalpha / ...
%%\bibliography{literature} %You need a file 'literature.bib' for this.
%
%%% The List of Figures
%\clearpage
%\addcontentsline{toc}{chapter}{List of Figures}
%\listoffigures
%
%%% The List of Tables
%\clearpage
%\addcontentsline{toc}{chapter}{List of Tables}
%\listoftables
%
%
%%%%%%%%%%%%%%%%%%%%%%%%%%%%%%%%%%%%%%%%%%%%%%%%%%%%%%%%%%%%%%
%%% APPENDICES
%%%%%%%%%%%%%%%%%%%%%%%%%%%%%%%%%%%%%%%%%%%%%%%%%%%%%%%%%%%%%%
%\appendix
%%% ==> Write your text here or include other files.
%
%%\input{FileName} %You need a file 'FileName.tex' for this.


\end{document}

