\documentclass[letterpaper,twoside,10pt]{article}

%% Language %%%%%%%%%%%%%%%%%%%%%%%%%%%%%%%%%%%%%%%%%%%%%%%%%
%\usepackage[USenglish]{babel} %francais, polish, spanish, ...
%\usepackage[T1]{fontenc}
%\usepackage[ansinew]{inputenc}
%\usepackage{lmodern} %Type1-font for non-english texts and characters

%% Packages for Graphics & Figures %%%%%%%%%%%%%%%%%%%%%%%%%%
\usepackage{graphicx} %%For loading graphic files

%% Math Packages %%%%%%%%%%%%%%%%%%%%%%%%%%%%%%%%%%%%%%%%%%%%
\usepackage{amsmath}
\usepackage{amsthm}
\usepackage{amsfonts}
\usepackage[top=1in, bottom=1.25in, left=1.25in, right=1.25in]{geometry}


\newcommand{\Z}{\mathbb{Z}} 

\begin{document}

Proof. The square of any odd number is also odd. 
\vspace{.5cm}

Sketch of the proof. It is straightforward to demonstrate that an odd number times itself is odd. However, if we are not willing to accept out of hand that a number $x$ is odd if and only if $x$ can be written in the form $2a+1$ for some integer $a$, then the proof is no longer trivial. This assertion relies on the division algorithm, which in turn relies on the well-ordering axiom. This serves to remind the reader that spaces which cannot be ordered, such as the complex numbers, cannot define the property of parity, but also that other more abstract spaces which can be ordered, such integer coordinates in Euclidean n-spaces, do indeed support an analogous concept of parity.

\begin{enumerate} 
	% Step 1: what exactly do we need to assume?
	\item It suffices to assume that for an odd number $x$, 
	% List of assumptions
	\begin{enumerate}
		\item if $x$ is not an even number, then $x$ is an odd number 
		\item $x$ can be represented as $2k$ for some $k \in \mathbb{Z}$, if and only if $x$ is an even number
		\item The division algorithm: For any $a$, $b\in\mathbb{Z}$, $\exists !$ $q \in \mathbb{Z}$, and $\exists !$ $r \in \mathbb{Z}$, with $0\leq r < b$ such that $x = b\cdot q + r$. (for which it suffices to assume the well-ordering principle, which states that every non-empty set of positive integers contains a least element)
	\end{enumerate}
	% Step 2: show that x can be represented as 2a+1
	\item There exists an $a \in \mathbb{Z}$ such that $x = 2a+1$
		% Proof of step 2
		\begin{enumerate}
			\item If $x$ is odd then by 1, $x$ is not even, so there does not exist a $m \in\Z$ such that $x=2m$. To show that there exists some $n \in\Z$ such that $x=2n+1$, assume for a contradiction that there does not exist an $n\in\Z$ such that $x=2n+1$. By the division algorithm, for $b=2$, then there exists $q\in\Z$ such that either $x=2q$ or $x=2q+1$. Therefore, by contradiction, there must exist some $n\in\Z$ such that $x=2n+1$ if $x$ is odd. 
		\end{enumerate}
	% Step 3: show that x^2 is also odd
	\item $x^2 = 2a' + 1$, where $a' = (2a^2 + 2a)$
		%Proof of step 2
		\begin{enumerate}
			\item Multiply each term and add like terms (a.k.a. "FOIL" or "First, outer, inner, last"): 
				\begin{align*}
					x^2 &= x \cdot x \\
					&= (2a+1)(2a+1)\\
					&= (2a)(2a) + (2a)(1) + (2a)(1) + (1) \\
					&= 4a^2 + 2a + 2a + 1 \\
					&= 4a^2 + 4a + 1 \\
					&= 2(2a^2 + 2a) + 1 \\
					&= 2a' + 1 \text{ where } a' = 2a^2 + 2a
				\end{align*}
		\end{enumerate}
	%Step 3
	\item QED
		% Show that if a is an integer then 2a^2 + 2a is also an integer
		% Show that if a number can be represented as 2a'+1 then that number is odd.  
		\begin{enumerate}
			\item For all $a, b \in\mathbb{Z}$ and $n \in \mathbb{N}$, then $a+b \in \mathbb{Z}$, $a\cdot b \in \mathbb{Z}$, and $a^n \in \mathbb{Z}$, therefore, $a' \in \mathbb{Z}$. 
			\item Any number which can be represented as $2m+1$ where $m\in\Z$ is not even, therefore that number is odd. Assume for a contradiction that $x=2m+1$ and that $x$ is also even, that is that $x=2n$ for some $n\in\Z$ Then:
			\begin{align*}
			2m+1 &=2n\\
			1 &= 2n-2m\\
			1 &= 2(n-m)
			\end{align*}
			Because $n-m \in\Z$, this implies that 2 divides 1, which is a contradiction. Therefore, we can conclude that any number which can be written as $2m+1$ must be odd. 
		\end{enumerate}
\end{enumerate}


\end{document}

